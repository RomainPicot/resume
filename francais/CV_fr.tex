\documentclass[10pt]{article}
\usepackage[utf8]{inputenc}

\usepackage{marginnote}
\usepackage{graphicx}
\usepackage{array}

\usepackage{geometry}
\usepackage{array,etaremune,microtype,pifont}
\usepackage{ragged2e,titlesec,xcolor}

\usepackage{hyperref}

\hypersetup
{
  hidelinks = true,
  pdfauthor = Romain Picot,
  pdftitle  = CV
}

\geometry
{
  a4paper,
  nohead,
  nofoot,
  hmargin = 1.5cm,
  vmargin = 2  cm,
}

\titleformat{\section}{\Large\bfseries\sffamily}{}{0em}
{%
  \begingroup
  \color{gray!30}%
  \titleline{\leaders\hrule height 0.6em\hfill\kern 0 pt\relax}%
  \endgroup
  \nobreak
  \vspace{-1.2em}%
  \nobreak
}

\titleformat{\subsection}{\large\itshape}{}{0em}{}

\renewcommand*\arraystretch{1.4}
\pagestyle{empty}
\frenchspacing

\newcommand*{\paper}[2]
{\item \href{http://dx.doi.org#1}{\ignorespaces#2\unskip.}}
\newcommand*{\papertitle}[1]
{%
  \begingroup
  \emph{#1}%
  \endgroup
}

\newlength{\sidewidth}
\newlength{\sidewidthT}
\newlength{\mainwidth}
\newlength{\mainwidthT}
\AtBeginDocument
{%
  \settowidth{\sidewidth}{\textbf{Professional bodiesAAAAA}\hspace{0.75em}}%
  \setlength{\mainwidth}{\dimexpr\linewidth - \sidewidth\relax}%
  \settowidth{\sidewidthT}{\textbf{Associatif}\hspace{1.em}}%
  \setlength{\mainwidthT}{0.5\dimexpr\linewidth - \sidewidthT\relax}%
}

\newcommand*{\headline}[1]
{%
\hbox{%
  \llap{\ding{72}\hspace*{0.2em}}%
  \textbf{#1}%
  }%
}

\newenvironment{CVtable}
{%
  \begin{tabular}
    {@{}>{\bfseries}p{\sidewidth}@{}>{\ding{229} \RaggedRight}p{\mainwidth}@{}}%
}
{\end{tabular}}

\newenvironment{CVtable3}
{%
  \begin{tabular}
    {@{}>{\bfseries}p{\sidewidthT}@{}>{\ding{229} }p{\mainwidthT}@{}>{\RaggedRight}p{\mainwidthT}@{}}%
}
{\end{tabular}}

\begin{document}


\begin{raggedright}
\Huge Doctorant \\
\LARGE 2\textsuperscript{ème} année
\end{raggedright}

\vspace{-2cm}
\begin{raggedleft}
  \textbf{Romain Picot}\\
  15 Rue René Anjolvy \\
  94250 Gentilly \\
  Tel: 06 71 86 39 81 \\
  \href{mailto:romain.picot2@gmail.com}{\texttt{romain.picot2@gmail.com}}\\
\end{raggedleft}

\section{Expérience professionnelle}

\begin{CVtable}
  Avr 2015 -- Mars 2018 & \textbf{Doctorant -- EDF R\&D, département
    SINETICS et LIP6,}\par\textbf{équipe PEQUAN}\par
  Vérification numérique des codes de calculs industriels\\
  Déc 2014 -- Fév 2015 &  \textbf{Ingénieur d'étude -- LIP6, équipe PEQUAN}\par Validation numérique des algorithmes compensés grâce à l'arithmétique\par stochastique \\
  Mars 2014 -- Sept 2014 &  \textbf{Stagiaire -- EDF R\&D, département SINETICS}\par Création d'une d'un outil de post-traitement à la librairie CADNA et\par utilisation sur le code Telemac~2D\\
  Juin 2013 -- Août 2013 &  \textbf{Stagiaire -- Armines}\par Conception et réalisation d'un disjoncteur intelligent pour plates-formes expérimentales \\
  Juin 2012 -- Sept 2012 & \textbf{Stagiaire WCM -- Saint-Gobain
    Belgium division Gyproc}\par
  Ingénierie des procédés\\
\end{CVtable}

\section{Formation}

\begin{CVtable}
2011 -- 2014 &  \textbf{Diplôme d'ingénieur -- Polytech Paris-UPMC}\par
               Électronique et informatique des systèmes embarqués \\
2009 -- 2011 &  \textbf{Classe préparatoire Mathématique et Physique -- Lycée Jacques Amyot}\par MPSI-MP, Melun\\
2006 -- 2009 &  \textbf{Baccalauréat Scientifique-- Lycée Joliot Curie}\par Option physique-chimie, Dammarie-les-lys
\end{CVtable}
 
\section{Langages}
\begin{CVtable}
Anglais & \textbf{TOEIC: 965/990 en 2013} \\
\end{CVtable}

\section{Informatique}
\begin{CVtable}
Système d'exploitation & Windows, Linux, Unix \\
Langage & C/C++, html/css, LaTeX, Fortran, Python \\
Autre & Inkscape, matlab, Emacs, MS Office, LibreOffice
\end{CVtable}


\section{Divers}
\begin{CVtable3}
Associatif & \textbf{AIPPU} \par Association des Ingénieurs Polytech Paris-UPMC & Membre du CA (2015 - .. ) \par
                      Vice-président (2016 - .. )\\ 
Sport & \textbf{Course} & \\
      & \textbf{Escrime médiévale} & \\
Other & \textbf{Lecture} & \\
      & \textbf{Jeux de plateaux} & \\
\end{CVtable3}

\section{Publications}
        \begin{itemize}
               \item[] {\bf Conférence internationale (avec comité de lecture):}
                 
               \begin{itemize}
               \item[$\bullet$] \emph{Numerical validation of
                   compensated summation algorithms with stochastic
                   arithmetic.}, S. Graillat, F. Jézéquel, et
                 R. Picot.  8th International Workshop on Numerical
                 Software Verification, NSV 2015, Avril
                 2015. Electronic Notes in Theoretical Computer
                 Science, volume 317, pages 55-69.
               \end{itemize}
                 
               \item[] {\bf Conférence internationale (avec résumé seul):}
                 
               \begin{itemize}
               \item[$\bullet$] \emph{PROMISE: floating-point
                   precision tuning with stochastic arithmetic},
                 S. Graillat, F. Jézéquel, R. Picot, F. Févotte et
                 B. Lathuilière. 17th International Symposium on
                 Scientific Computing, Computer Arithmetics and
                 Verified Numerics, SCAN 2016, Septembre 2016
               \end{itemize}
                 
               \item[] {\bf Rapport de recherche:}
                 
               \begin{itemize}
               \item[$\bullet$] \emph{Auto-tuning for floating-point
                   precision with Discrete Stochastic Arithmetic},
                 S. Graillat, F. Jézéquel, R. Picot, F. Févotte et
                 B. Lathuilière. Référence: hal-01331917
               \end{itemize}
                 
               \item[] {\bf Présentations:}
                 
               \begin{itemize}
                 \item[$\bullet$] \emph{Arithmétique stochastique et Algorithmes compensés}, Séminaire des Doctorants du LIP6, Février 2016
                 \item[$\bullet$] \emph{Algorithmes compensés et Arithmétique stochastique}, Journée des thèses SINETICS, Avril 2016
               \end{itemize}
                                    
       \end{itemize}


\end{document}